\documentclass[10pt]{article}
\usepackage[spanish]{babel}
\usepackage[a4paper, tmargin=0.75in, lmargin=0.80in, rmargin=0.80in, bmargin=1in]{geometry}
\usepackage{hyperref}
%\usepackage{multicol}
\hypersetup{
    colorlinks=true,
    linkcolor=black,
    filecolor=magenta,      
    urlcolor=blue,
    citecolor=black,
}
%\usepackage[numbers,sort&compress]{natbib} % for a numerical citation list
\usepackage{natbib} % to cite references by surname and year
\usepackage{graphicx}
\usepackage{amsfonts}
\usepackage{amsthm}
\usepackage{amssymb}
\usepackage{lipsum}
\usepackage{amsmath}
\usepackage{tabularx}
\usepackage{pdflscape}
\usepackage{booktabs}
\usepackage{bbm}
\usepackage{listings}
\usepackage{xcolor}

\usepackage[
  backend=biber,
  style=alphabetic,
  citestyle=alphabetic,
  doi=true,
  url=true,
  isbn=false,
  eprint=false,
  maxbibnames=99
]{biblatex}

\addbibresource{referencias.bib}

% Definimos colores estilo "terminal con fondo negro"
\definecolor{backcolour}{rgb}{0.1,0.1,0.1}
\definecolor{codegreen}{rgb}{0,0.8,0}
\definecolor{codegray}{rgb}{0.7,0.7,0.7}
\definecolor{codepurple}{rgb}{0.8,0.6,1}
\definecolor{codewhite}{rgb}{1,1,1}


\lstdefinestyle{mypython}{
    backgroundcolor=\color{backcolour},
    basicstyle=\ttfamily\scriptsize\color{codewhite},
    commentstyle=\color{codegreen},
    keywordstyle=\color{codepurple},
    stringstyle=\color{codegreen},
    numbers=left,
    numberstyle=\tiny\color{codegray},
    breaklines=true,
    breakatwhitespace=false,
    showstringspaces=false,
    tabsize=4
}

\lstset{style=mypython}







\pagestyle{empty}


%%%%%%%%%%%%%%%%%%%%%%%%%%%%%%%%%%%%%%%%%%%%%%%%%%
%%%%%%%%%%%%%%%%%%%%%%%%%%%%%%%%%%%%%%%%%%%%%%%%%%
%%%%%%%%%%%%%%%%%%%%%%%%%%%%%%%%%%%%%%%%%%%%%%%%%%
%%%%%%%%%%%%%%%%%%%%%%%%%%%%%%%%%%%%%%%%%%%%%%%%%%
% ENTER SOME IMPORTANT INFORMATION
%%%%%%%%%%%%%%%%%%%%%%%%%%%%%%%%%%%%%%%%%%%%%%%%%%
%%%%%%%%%%%%%%%%%%%%%%%%%%%%%%%%%%%%%%%%%%%%%%%%%%
%%%%%%%%%%%%%%%%%%%%%%%%%%%%%%%%%%%%%%%%%%%%%%%%%%
%%%%%%%%%%%%%%%%%%%%%%%%%%%%%%%%%%%%%%%%%%%%%%%%%%
\newcommand{\studentname}{Sánchez, Hazel; Hernández, Deany ; Canché, Elías}
\newcommand{\researchcentre}{Maestría en Probabilidad y Estadística}
\newcommand{\institution}{Centro de Investigación en Matemáticas (CIMAT)}
\newcommand{\projecttitle}{Tarea 2}
\newcommand{\supervisor}{Dr. Marco Antonio Aquino López}
%%%%%%%%%%%%%%%%%%%%%%%%%%%%%%%%%%%%%%%%%%%%%%%%%%
%%%%%%%%%%%%%%%%%%%%%%%%%%%%%%%%%%%%%%%%%%%%%%%%%%
%%%%%%%%%%%%%%%%%%%%%%%%%%%%%%%%%%%%%%%%%%%%%%%%%%
%%%%%%%%%%%%%%%%%%%%%%%%%%%%%%%%%%%%%%%%%%%%%%%%%%
%%%%%%%%%%%%%%%%%%%%%%%%%%%%%%%%%%%%%%%%%%%%%%%%%%
%%%%%%%%%%%%%%%%%%%%%%%%%%%%%%%%%%%%%%%%%%%%%%%%%%

\begin{document}

\begin{center}
{\Large{Proyecto 2: Clasificación Supervisada}} \\
\vspace{2mm}
{\Large{Introducción a la Ciencia de Datos}} \\
\end{center}

\vspace{5mm}
\hrule
\vspace{1mm}
\hrule

\vspace{3mm}
\begin{tabular}{ll} 
Integrantes:           	        & {\studentname}   \\ 
Programa Educativo: 	        & {\researchcentre}  \\ 
Institución:                 & {\institution}  \\
Profesor: 	                 & {\supervisor}  \\ 
\end{tabular}

\vspace{3mm}
\hrule
\vspace{1mm}
\hrule

\begin{abstract}
\lipsum[1]
\end{abstract}

\section{Introducción}
Aquí va una introducción


\section{Descripción y exploración inicial de los datos}\label{sec:dataset}
Para este trabajo se utilizó la base de dattos \textbf{Bank Marketing} (\texttt{bank-full.csv}), la cual contiene información sobre clientes de una entidad bancaria y el resultado de campañas de marketing telefónicas. 

En primer lugar, se realizó una inspección general de la base. El archivo cuenta con \textbf{45,211 observaciones} y \textbf{17 variables pedictoras}, más la variable de respuesta \textbf{y}, que indica si el cliente aceptó (\texttt{yes}) o no (\texttt{no}) contratar el depósito a plazo.   


\subsection{Descripcis}
Hola mundo



\section{Preprocesamiento}


\section{Manejo de datos faltantes y de \textit{outliers}}
\subsection{Datos faltantes}\label{subsec:missing}



\subsection{\textit{Outliers}}

\section{Conclusiones}









\end{document}
